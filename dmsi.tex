\documentclass[12pt,a4paper,twoside]{report}
%%%%%%%%%%%%%%%%%%%%%%%%%%%%%%%%%%%%%%%%
%\usepackage{tocbibind} %para incluir a Lista de Figuras, Lista de Tabelas no Índice
%%%%%%%%%%%%%%%%%%%%%%%%%%%%%%%%%%%%%%%%
\usepackage{float}
\floatstyle{plain}
\newfloat{Code Excerpt}{}{program}[chapter]
\newfloat{Program}{}{program}[chapter]
%%%%%%%%%%%%%%%%%%%%%%%%%%%%%%%%%%%%%%%%
\newcommand{\R}{{\mbox{I$\!$R}}}
\usepackage{style/relatorio} % formato "oficial"
\usepackage{graphicx}
\graphicspath{{./images/}}
%%%%%%%%%%%%%%%%%%%%%%%%%%%%%%%%%%%%%%%%
\usepackage{fancyvrb} %para poder utilizar função especial do Verbatim
\usepackage{pdfpages} %para poder incluir pdf's externos
%%%%%%%%%%%%%%%%%%%%%%%%%%%%%%%%%%%%%%%%
\usepackage{verbatim} % incluir listagens
\usepackage{listings} % incluir listagens
\usepackage{url} % typeset URL's
\usepackage{textcomp}
\usepackage{multirow}
\usepackage{epstopdf}
\usepackage{chngcntr} % para permitir que o footnote tenha numeraçao incrementada (ver: \counterwithout)
%\usepackage{fancyhdr}
%\usepackage{hhline}
\usepackage[pdftex,colorlinks=true,
                    urlcolor=black, %blue
                    linkcolor=black,
                    citecolor=black, %cor das citações
                    bookmarks=true,
                    pdfstartview=FitH]{hyperref}

%%%%%%%%%%%%%%%%%%%%%%%%%
\usepackage{amsmath}
\usepackage{amssymb}
\newcommand{\sign}{\operatorname{sgn}}
%\newcommand{\sign}{\mathop{\mathrm{sgn}}}
\usepackage[english,ruled]{algorithm2e}
\parskip=0.25cm
\usepackage{boxedminipage}
\usepackage{epigraph}
\usepackage{lipsum}
\usepackage{style/slashbox} %% poder usar barras diagonais nas tabelas
%%%%%%%%%%%%%%%%%%%%%%%%%

\setstretch{1.5}
%%%%%%%%%%%%%%%%%%%%%%%%%
\setlength\epigraphwidth{10cm}
\setlength\epigraphrule{0pt}

\begin{document}
 %\cleardoublepage
%\dept{Engenharia Informática - 3º Ano}
%\course{Engenharia Informática}
\counterwithout*{footnote}{chapter} %garante que a numeração do footnote incrementa (comentar isto se quiser que a numeração vá a 1 em cada novo capitulo)
\title{Titulo da Documento}
\author{Autor do documento}
\authnum{}
\secondauthor{\mbox{}}
\secauthnum{\mbox{}}
\supervisor{Nome do supervisor}
%\cosupervisor{\mbox{Nome do co-supervisor}}
%\firstreader{\mbox{}}

% Coloca a capa, primeira pagina e outros
\beforepreface
%cria uma página em branco
%\newpage
%\mbox{}

\hfill Aos .....,


\newpage
\mbox{}

\hfill citaçao famosa

\prefacesection{Resumo}
este e o resumo

\lipsum[1]

\newpage
\mbox{}


\prefacesection{Abstract}
This is abstract

\lipsum[1]


\newpage
\mbox{}

\prefacesection{Agradecimentos}
o blabla bla dos agradecimentos!
\lipsum[1]

% Coloca indices
\afterpreface

%\newpage
%\mbox{}

\prefacesection{Lista de Abreviaturas}
\begin{description}
 \item [AMD] Advanced Micro Devices
 \item [API] Application Programming Interfaces
 \item [APP] Accelerated Parallel Processing
 \item [APU] Accelerated Processing Unit
 \item [ASIC] Application-Specific Integrated Circuit
 \item [ATLAS] Automatically Tuned Linear Algebra Software
 \item [BLAS] Basic Linear Algebra Subprograms
 \item [BSD] Berkeley sockets
 \item [CU] Compute Unit
 \item [C99] C Programming Language Standard
 \item [CPU] Central Processing Unit
 \item [CUDA] Compute Unified Device Architecture
 \item [DSP] Digital Signal Processor
 \item [FPGA] Field-Programmable Gate Array
 \item [GPP] General Purpose Processors
 \item [GPU] Graphics Processing Unit
 \item [GPGPU] General-Purpose Graphics Processing Units
 \item [GVirtuS] Generic Virtualization Service
 \item [HPC] High-Performance Computing
 \item [IO] Input Output
 \item [ISA] Instruction Set Architecture
 \item [ISO] International Organization for Standardization
 \item [JIT] Just-In-Time
 \item [MPI] Message Passing Interface
 \item [OS] Operative System
 \item [OpenCL] Open Computing Language
 \item [OpenMP] Open Multi-Processing
 \item [PE] Processing Element
 \item [PVM] Parallel Virtual Machine
 \item [RAM] Random Access Memory
 \item [SDK] Software Development Kit
 \item [SMP] Symetric MultiProcessing
 \item [SIMD] Single Instruction, Multiple Data
 \item [SPMD] Single Program, Multiple Data
 \item [SSH] Secure-Shell
 \item [VCL] Virtual OpenCL
\end{description}

\bodystart

% Capitulos do documento
\chapter{Introdução}\label{cap:intro}

Para ver como são os acrónimos, este documento é elaborado na \gls{ESTiG}.

\lipsum[1-3]

\begin{figure}[htbp]
    \begin{center}
    \includegraphics[scale=0.05]{images/imagem01}
    \end{center}
    \caption{exemplo de imagem 1.}
    \label{fig:graficosubscricoesmoveis}
\end{figure}


%\input{chapters/cap02}
%\input{chapters/cap03}
%\input{chapters/cap04}
\chapter{Conclusão}\label{cap:conclu}

\lipsum[1-3]





% estilo de referências. outros valores posíveis são 'plain' e 'abbrv' apalike
\bibliographystyle{plain}
% listagem de referências
\bibliography{lib/refs}


% Apêndice
\appendix
\chapter{Apendice Geral}
\label{apendice1}

\lipsum[1-3]

\end{document}
