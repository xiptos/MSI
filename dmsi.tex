\documentclass[12pt,a4paper,twoside]{ipb}

% comentar caso seja uma disseração de mestrado
\usepackage{projei}

\usepackage[portuguese]{babel}
\graphicspath{{./images/}}
\usepackage{listings} % incluir listagens
\usepackage{url} % typeset URL's
\usepackage[pdftex,colorlinks=true,
                    urlcolor=black, %blue
                    linkcolor=black,
                    citecolor=black, %cor das citações
                    bookmarks=true,
                    pdfstartview=FitH]{hyperref}

\usepackage{lipsum}

%%%%%%%%%%%%%%%%%%%%%%%%%

\title{Título do Documento}
\author{Autor do documento}
%\authnum{}
%\secondauthor{}
%\secauthnum{}
\supervisor{Nome do orientador}
%\cosupervisor{Nome do co-orientador}
%\firstreader{Nome do arguente}
% Para definir o ano letivo
%\courseyear{1234}
% Para nao mostrar a lista de tabelas
\tablespagefalse

\makeglossaries
\loadglsentries{acronym}

\begin{document}

% Coloca a capa, primeira pagina e outros
\beforepreface

\cleardoublepage
\prefacesection{Dedicatória}
Aos \ldots
\cleardoublepage

\prefacesection{Resumo}
\lipsum[1]

\cleardoublepage

\prefacesection{Abstract}
\lipsum[1]

\cleardoublepage

\prefacesection{Agradecimentos}
\lipsum[1]

% Coloca indices
\afterpreface

\printglossary[type=\acronymtype,title={Acrónimos}]

\bodystart

% Capitulos do documento
\chapter{Introdução}\label{cap:intro}

Para ver como são os acrónimos, este documento é elaborado na \gls{ESTiG}.

\lipsum[1-3]

\begin{figure}[htbp]
    \begin{center}
    \includegraphics[scale=0.05]{images/imagem01}
    \end{center}
    \caption{exemplo de imagem 1.}
    \label{fig:graficosubscricoesmoveis}
\end{figure}


%\input{chapters/cap02}
%\input{chapters/cap03}
%\input{chapters/cap04}
\input{chapters/conclusoes}

% estilo de referências. outros valores posíveis são 'plain' e 'abbrv' apalike
\bibliographystyle{plain}
% listagem de referências
\bibliography{lib/refs}


% Apêndice
\appendix
\input{chapters/apendice}

\end{document}
